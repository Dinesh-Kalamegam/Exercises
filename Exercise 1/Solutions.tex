\documentclass[10pt]{article}
\usepackage[left=3cm,right=3cm,top=3cm,bottom=3cm]{geometry}
\usepackage{amsmath}
\usepackage{amsthm}
\usepackage{amssymb}
\usepackage{bm}
\usepackage{tabularx}

\begin{document}
\title{MATH7501: Exercise 1 Solutions}
\author{Dinesh Kalamegam}
\date{\today}
\maketitle

\numberwithin{equation}{subsection}
\newtheorem{theorem}{Theorem}[section]
\newtheorem{definition}[theorem]{Defintion}
\newtheorem{proposition}[theorem]{Proposition}
\newtheorem{corollary}[theorem]{Corollary}

\section{Question 1 (6 MARKS)}
\subsection{Find $E(X)$}
The possible outcomes are $\{TT,TH,HT,HH\}$ $X$ is the number of heads so here we have $\{0,1,1,2\}$ as the values of X following the order of the outcomes in the previous set
\\
The probability mass function (\emph{pmf}) is then given by:
\\

\begin{tabularx}{\textwidth}{ |X|X|X|X| }
  \hline
  $k$ & $0$ & $1$ & $2$ \\
  \hline
  $P(X=k$) & $1/4$  & $1/2$  & $1/4$  \\
  \hline
\end{tabularx}
\\
\\
Then
\begin{align*}
    E(X) &= \sum_{k} kP(X=k) \\
         &= 0 \cdot P(X=0) + 1 \cdot P(X=1) + 2 \cdot P(X=2) \\
         &= (0 \cdot 1/4) + (1 \cdot 1/2) + (2 \cdot 1/4) \\
         &= 1
\end{align*}
So we have that $E(X)=1$

\subsection{Find $E(1/1+X)$}
Following a similar process:
\begin{align*}
    E(1/1+X) &= \sum_{k} \frac{1}{1+k} P(X=k) \\
         &= 1 \cdot P(X=0) + 1/2 \cdot P(X=1) + 1/3 \cdot P(X=2) \\
         &= (1 \cdot 1/4) + (1/2 \cdot 1/2) + (1/3 \cdot 1/4) \\
         &= \frac{7}{12}
\end{align*}

\subsection{Verify that $E(\frac{1}{1+X}) \neq \frac{1}{1+E(X)}$ }
We just got $E(\frac{1}{1+X})$ in \textbf{section 1.2} $ = \frac{7}{12}$ and in \textbf{section 1.1} $E(X)=1$ so:
\begin{align*}
    \frac{1}{1+E(X)} &= \frac{1}{1+1}\\
                     &= \frac{1}{2}\\
                     &\neq \frac{7}{12} \\
                     &\neq E(\frac{1}{1+X})\\
                     & \text{\textbf{As required}}\\
\end{align*}

\subsection{Find $Var(1/1+X)$}
To find this compute :

\begin{equation}
    E((1/1+X)^2) - (E(1/1+X))^2
\end{equation}
\\
First Compute
\begin{align*}
    E((1/1+X)^2) &= (1 \cdot (1/4)^2) + ((1/2)^2 \cdot (1/2)) + ((1/3)^2 \cdot (1/4)) \\
                 &= \frac{29}{72}
\end{align*}
\\
So back to \textbf{Equation 1.4.1}
\begin{align*}
    E((1/1+X)^2) - (E(1/1+X))^2 &= \frac{29}{72} - (\frac{7}{12})^2 \\
                                &= \frac{1}{16}
\end{align*}

\section{Question 2 (5 MARKS )}
\subsection{Find possible values of X and corresponding pmf}
X is our winnings with Red worth -1, White worth 0 , Blue worth +2.
We have 5 Red (R), 3 Blue (B) and 2 White (W). \\
Then we have the possible events
\bm{$ \{RR,RW,WR,WW,RB,BR,WB,BW,BB\} $}
Then values of X are  \bm{$\{-2,-1,-1,0,1,1,2,2,4\}$}
Let us now find the probability mass function. It is given by: \\
$
    P(X=-2)= P(\{RR\}) = (\frac{5}{10}) \times (\frac{4}{9}) = \frac{20}{90} \\
$
\\
$
    P(X=-1)= P(\{RW, WR\}) = 2\times(\frac{5}{10}) \times (\frac{2}{9}) = \frac{20}{90} \\
$
\\
$
    P(X=0) = P(\{WW\}) = (\frac{2}{10}) \times (\frac{1}{9}) = \frac{2}{90} \\
$
\\
$
   P(X=1) = P(\{RB,BR\})= 2\times(\frac{5}{10}) \times (\frac{3}{9}) = \frac{30}{90} \\
$
\\
$
   P(X=2) = P(\{WB,BW\})= 2\times(\frac{2}{10}) \times (\frac{3}{9}) = \frac{12}{90} \\
$
\\
$
    P(X=4) = P(\{BB\})= (\frac{3}{10}) \times (\frac{2}{9}) = \frac{6}{90} \\
$

\subsection{Expected Profit}
The Expected Profit can be found by calculating $E(X)$
\begin{align*}
    E(X) &= \sum_{k} kP(X=k) \\
         &= (-2 \cdot P(X=-2)) + ((-1) \cdot P(X= -1)) + ...+ (4 \cdot P(X=4)) \\
         &= 0.2
\end{align*}
So in terms of the context we can say that the Expected Profit is \pounds 0.20

\subsection{Probability that you lose a pound given that you make a loss}
\emph{Recall:} $ P(A|B) = \frac{P(A \cap B)}{P(B)}$ conditional probability \\
\begin{align*}
    P(X < -1 | X < 0) &= \frac{P((X < -1) \cap (X < 0))} {P(X<0)} \\
                      &= \frac{P(X < -1)}{P(X < 0)}\\
                      &= \frac{P(X = -2)}{P(X=-2) + P(X=-1)}\\
                      &= \frac{20/90}{20/90 + 20/90} \\
                      &= \frac{1}{2}
\end{align*}
\section{Question 3 (4 MARKS)}
\subsection{Calculate probability that at most a proportion $\alpha = k/n$ of the organisms survive }
\begin{equation*}
    P(X=r) = \frac{2(r+1)}{(n+1)(n+2)}  \text{where $r = 0,1,...,n$}
\end{equation*} \\
"At most $k$ out of $n$ organisms survive" and the corresponding probability:\\
\begin{align*}
    &= \frac{2}{(n+1)(n+2)} \cdot \frac{(k+1)(k+2)}{2} \\ \\
    &= \frac{(k+1)(k+2)}{(n+1)(n+2)}\\ \\
\end{align*}

\subsection{Deduce that for large $n$ this probability is approximately $\alpha^2$}

\begin{align*}
    P(X\leq k) &= \frac{(\frac{k}{n} + \frac{1}{n})(\frac{k}{n} + \frac{2}{n})}{(1 + \frac{1}{n})(1 + \frac{2}{n})} \\ \\
    &= \frac{(\alpha + \frac{1}{n})(\alpha + \frac{2}{n})}{(1 + \frac{1}{n})(1 + \frac{2}{n})}
\end{align*}
We can now find $\lim_{n \to \infty}$ which would be
\begin{align*}
        \frac{(\alpha + 0)(\alpha + 0)}{(1 + 0)(1 + 0)}
        &= \alpha^2
\end{align*}
As required

\subsection{Find the smallest value of $n$ for which the probability of there being at least one survivor among the $n$ organisms is at least 0:95.}
The required probability for this final part is \\
i.e.
\begin{align*}
    P(X\geq 1) \geq 0.95  &\iff P(X=1) + ... + P(X=n) \geq 0.95 \\
                          &\iff P(X<1) \leq 0.05 \\
                          &\iff P(X=0) \leq 0.05 \\
                          &\iff \frac{2}{(n+1)(n+2)} \leq 0.05 \\
                          &\iff (n+1)(n+2) \geq 40 \\
\end{align*}
Now the smallest value of $n$ that satisifies this would be \bm{$n=5$}. This is because $n=5$ is the first value of $n$ s.t. $(n+1)(n+2) \geq 40$ [$(5)(6)\leq 40$ but $(6)(7) \geq 40$] \\ \\
So the answer is \bm{$n=5$}


\end{document}
